\documentclass{beamer}
\usepackage[T1]{fontenc}
\usepackage{fancyvrb}
\usepackage{minted}

\title{Ray Marching}
\author{John Owen}
\date{}

\newcommand\graphic[1]{\includegraphics[keepaspectratio=true,width=\textwidth]{../img/#1}}
\newcommand\imageframe[2]{
    \begin{frame}
        \frametitle{#1}
        \includegraphics[keepaspectratio=true,width=0.95\textwidth]{../img/#2}
    \end{frame}
}

\begin{document}

\frame{\titlepage}


\begin{frame}[fragile]
    \frametitle{Distance functions}
    Ray marching algorithms are rendering algorithms like rasterization or ray
    tracing but instead of triangles and shape-ray intersection functions, ray
    marching algorithms use distance functions, which take a point `p` and
    return the distance of `p` to the boundary of the shape. For example, the
    distance function of a sphere is:
    \begin{minted}[fontsize=\footnotesize]{glsl}
float sphere(vec3 p, float radius) {
  return length(p) - radius;
}
    \end{minted}
    \graphic{looking-outside-circle.png}
\end{frame}

\imageframe{Sphere Tracing Algorithm}{raymarch1.png}
\imageframe{Sphere Tracing Algorithm}{raymarch2.png}
\imageframe{Sphere Tracing Algorithm}{raymarch3.png}
\imageframe{Sphere Tracing Algorithm}{raymarch4.png}
\imageframe{Sphere Tracing Algorithm}{raymarch5.png}
\imageframe{Sphere Tracing Algorithm}{raymarch6.png}
\imageframe{Sphere Tracing Algorithm}{raymarch7.png}
\imageframe{Sphere Tracing Algorithm}{raymarch8.png}
\imageframe{Sphere Tracing Algorithm}{raymarch9.png}
\imageframe{Sphere Tracing Algorithm}{raymarch10.png}

\begin{frame}[fragile]
    \frametitle{Distance functions}
    \begin{minted}[fontsize=\small]{glsl}
float box(vec3 p, vec3 size) {
  return length(max(abs(p) - size / 2., 0.));
}

float sdfCylinder(vec3 p, float radius, float height) {
  vec2 h = vec2(radius, height / 2.);
  vec2 d = abs(vec2(length(p.yz), p.x)) - h;
  return min(max(d.x, d.y), 0.0) + length(max(d, 0.0));
}

float sdfPlane(vec3 p, vec3 normal, float distance) {
  return dot(p, normal) + distance;
}
    \end{minted}
\end{frame}

\begin{frame}[fragile]
    \frametitle{Sphere Tracing Algorithm}
    \begin{minted}[fontsize=\footnotesize]{glsl}
void main() {
  gl_FragColor = (0.7, 0.8, 0.9, 1.) // Start with a sky color
  vec3 ro = vec3(1., 0., 5.); // Ray origin
  vec3 rd = rayDirection();
  float t = 0.; // How far we've travelled along the ray so far
  for (int i = 0; i < 1000; i++) { // March 1000 times
    vec3 p = ro + rd * t; // The point along the ray we are at
    float d = sphere(p, 2.); // Distance of p to sphere of radius 2
    t += d; // Move forward along the ray
    if (d < 0.001) { // If p close to boundary sphere
      // Compute the lighting, shadows, etc
      gl_FragColor = computeShadingForPoint(p);
      break;
    }
    if (t > 100) // If the ray missed the sphere completely
      break;
  }
}
    \end{minted}
\end{frame}

\begin{frame}
    \frametitle{Sphere Tracing Algorithm}
    \graphic{ray-marching-overview.png}
\end{frame}

\begin{frame}[fragile]
    \frametitle{Translation}
    \begin{columns}
        \column{.5\textwidth}
        We can translate a shape by shifting the \texttt{p} value:

        \begin{minted}[fontsize=\footnotesize]{glsl}
// A sphere of radius 2 at (-1,1,0).
float scene(vec3 p) {
    p -= vec3(-1., 1., 0.);
    return sphere(p, 2.);
}
        \end{minted}
        \column{.5\textwidth}
        \graphic{translated-sphere.png}
    \end{columns}
\end{frame}

\begin{frame}[fragile]
    \frametitle{Rotation}
    \begin{columns}
        \column{.5\textwidth}
        A shape can be rotated by transforming \texttt{p} with some simple
        trigonometry:


        \begin{minted}[fontsize=\footnotesize]{glsl}
// Rotate around the y axis (yaw).
vec3 rotateY(vec3 p, float a) {
    float c = cos(a);
    float s = sin(a);
    float x = c * p.x + s * p.z;
    float z = -s * p.x + c * p.z;
    return vec3(x, p.y, z);
}

// A box with dimensions (2,1,1) rotated 45 degrees.
float scene(vec3 p) {
    p = rotateY(p, pi / 4.);
    return sbox(p, vec3(2., 1., 1.));
}
        \end{minted}
        \column{.5\textwidth}
        \graphic{rotated-box.png}
    \end{columns}
\end{frame}

\begin{frame}[fragile]
    \frametitle{Boolean Operations}
    \begin{columns}
        \column{.5\textwidth}
        \begin{minted}[fontsize=\footnotesize]{glsl}
// Box plus sphere.
float scene(vec3 p) {
    float d = box(p, vec3(4.0));
    d = min(d, sphere(p, 2.5));
    return d;
}
        \end{minted}
        \column{.5\textwidth}
        \graphic{box-plus-sphere.png}
    \end{columns}
\end{frame}

\begin{frame}[fragile]
    \frametitle{Boolean Operations}
    \begin{columns}
        \column{.5\textwidth}
        \begin{minted}[fontsize=\footnotesize]{glsl}
// Subtract shape with distance d2 from the shape with distance d1.
float opS(float d1, float d2) {
    return max(d1, -d2);
}

// Box minus a sphere.
float scene(vec3 p) {
    float d = box(p, vec3(4.0));
    d = opS(d, sphere(p, 2.5));
    return d;
}
        \end{minted}
        \column{.5\textwidth}
        \graphic{box-minus-sphere.png}
    \end{columns}
\end{frame}

\begin{frame}[fragile]
    \frametitle{Boolean Operations}
    \begin{columns}
        \column{.5\textwidth}
        \begin{minted}[fontsize=\footnotesize]{glsl}
// The shape that is the intersection of two shapes.
float opI(float d1, float d2) {
    return max(d1, d2);
}

// Intersection of box and sphere.
float scene(vec3 p) {
    float d = box(p, vec3(4.0));
    d = opI(d, sphere(p, 2.5));
    return d;
}
        \end{minted}
        \column{.5\textwidth}
        \graphic{box-intersect-sphere.png}
    \end{columns}
\end{frame}

\begin{frame}[fragile]
    \frametitle{Repetition}
    \begin{columns}
        \column{.5\textwidth}
        \begin{minted}[fontsize=\footnotesize]{glsl}
float scene(vec3 p) {
    float spacing = 10.;
    p = mod(p, spacing) -
        0.5 * spacing;
    return sphere(p, 0.5);
}
        \end{minted}
        \column{.5\textwidth}
        \graphic{infinite-spheres.png}
    \end{columns}
\end{frame}

\begin{frame}[fragile]
    \frametitle{Building Interesting Shapes}
    \graphic{building-interesting-shapes.png}
\end{frame}

\begin{frame}[fragile]
    \frametitle{Terrain Marching}
    \begin{columns}
        \column{.5\textwidth}
        \begin{minted}[fontsize=\footnotesize]{glsl}
// y=terrain(x,z)
float terrain(float x, float z) {
    return sin(x) * sin(z);
}
        \end{minted}
        \column{.5\textwidth}
        \graphic{sinxcosz.png}
    \end{columns}
\end{frame}

\imageframe{Terrain Marching Algorithm}{terrainmarch1.png}
\imageframe{Terrain Marching Algorithm}{terrainmarch2.png}
\imageframe{Terrain Marching Algorithm}{terrainmarch3.png}
\imageframe{Terrain Marching Algorithm}{terrainmarch4.png}
\imageframe{Terrain Marching Algorithm}{terrainmarch6.png}
\imageframe{Terrain Marching Algorithm}{terrainmarch7.png}
\imageframe{Terrain Marching Algorithm}{terrainmarch8.png}
\imageframe{Terrain Marching Algorithm}{terrainmarch9.png}
\imageframe{Terrain Marching Algorithm}{terrainmarch10.png}

\begin{frame}[fragile]
    \frametitle{Terrain Marching}
    \begin{minted}[fontsize=\footnotesize]{glsl}
void main() {
    gl_FragColor = vec4(0.8, 0.85, 9.); // Sky color
    vec3 ro = vec3(1., 0., 5.); // Ray origin
    vec3 rd = rayDirection(); // Ray direction
    float t = 0.;
    const float dt = 0.001;
    for (float t = 0.; t <= 1000.; t += dt) {
        vec3 p = ro + rd * t;
        if (p.y < terrain(p.x, p.z)) {
            t = t - 0.5 * dt;
            vec3 color = computeShadingForPoint(ro + rd * t);
            gl_FragColor = vec4(color, 1.);
        }
    }
}
    \end{minted}
\end{frame}

\imageframe{Terrain Marching}{terrain-marching.png}

\begin{frame}
    \frametitle{The Mandelbrot Fractal}
    \begin{columns}
        \column{.75\textwidth}
        Mandelbrot set is all values of $c$ where the sequence
        \begin{gather*}
            z_{n+1}=z_n^2+c,\\
            z_1=0,
        \end{gather*}
        doesn't diverge, for $z_n,c\in\mathbb{C}$. If we map
        \texttt{gl\_FragCoord.x} to $c$'s real part and \texttt{gl\_FragCoord.y}
        to $c$'s imaginary part we can visualize this set. Algorithm for
        drawing fractal:\\
       
        For each pixel mapped to $c$, compute $z_n$ for large $n$. If it
        diverges, color it white otherwise
        color it black.
        \column{.4\textwidth}
        \graphic{mandelbrot-escape-time.png}
    \end{columns}
\end{frame}

\begin{frame}
    \frametitle{The Mandelbrot Fractal: Improved}
    \begin{columns}
        \column{.7\textwidth}
        Improvement: Use distance functions. The distance from $c$ to the
        boundary of the Mandelbrot set is
        \[
            d = \lim_{n\to\infty}
                \frac{ \left|z_n\right| }{ \left|z'_n\right| }
                \frac{ \ln \left|z_n\right| }{ 2 }
        \]
        where $z'_n$ is the rate of change of $z$ at $n$. We can
        differentiate $z_n$ using the chain rule to get
        \begin{gather*}
            z'_{n+1} = 2z_nz'_n + 1,\\
            z'_1 = 0.
        \end{gather*}
        Now we can use the distance $d$ to darken the points close to the edge
        of the Mandelbrot set.
        \column{.3\textwidth}
        \graphic{mandelbrot-dist-func.png}
    \end{columns}
\end{frame}

\begin{frame}
    \frametitle{The Mandelbulb Fractal}
    \begin{columns}
        \column{.5\textwidth}
        Why restrict ourselves to 2D complex numbers?
        \[
            z=(z_x,z_y,z_z)
        \]
        \pause
        But how do we define $z^2$? Well, for a complex number $w=a+bi$,
        \[
            w^k = r^k(\cos(k\phi)+i\sin(k\phi))
        \]
        where
        \begin{gather*}
            r=\sqrt{a^2+b^2},\\
            \phi=\text{atan2}(b,a).
        \end{gather*}
        \column{.5\textwidth}
        i.e, convert it to polar coordinates, take the magnitude to a power of $k$
        and multiply the angle by $k$.
        \pause
        So for our triplex number $z$ we can do
        \begin{gather*}
            z_x^k = r^k\sin(k\theta)\cos(k\phi),\\
            z_y^k = r^k\sin(k\phi)\sin(k\theta),\\
            z_z^k = r^k\cos(k\theta).
        \end{gather*}
        where
        \begin{gather*}
            r=\sqrt{z_x^2+z_y^2+z_z^2},\\
            \phi=\text{atan2}(z_y,z_x),\\
            \theta=\text{acos}(z_z/r).
        \end{gather*}
    \end{columns}
\end{frame}

\imageframe{Mandelbulb $z_{n+1}=z_n^2+c$}{mandelbulb2.png}
\imageframe{Mandelbulb $z_{n+1}=z_n^8+c$}{mandelbulb8.png}

\end{document}
