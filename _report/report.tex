\documentclass[parskip=half]{scrartcl}
\usepackage{graphicx}
\usepackage{csquotes}
\usepackage{hyperref}
\usepackage{float}
\usepackage[backend=bibtex]{biblatex}
\usepackage{enumitem}
\usepackage[margin=1in]{geometry}
\usepackage{cleveref}

\hypersetup{colorlinks,urlcolor=blue}
\addbibresource{references.bib}
\MakeOuterQuote{"}
\setlength{\itemsep}{0em}
\newcommand\graphic[1]{%
    \centering
    \includegraphics[keepaspectratio=true,width=0.7\textwidth]{../img/#1}%
}
\crefname{figure}{figure}{figures}

\begin{document}

\section{Problem Statement}

The project is a website that teaches various basic and advanced techniques
based around the rendering technique known as ray marching. Using maths,
graphics, code snippets, user-modifiable shader programs, and plain English,
the website explains the implementation and some of the theory of:
\begin{itemize}
    \item The sphere tracing algorithm.
    \item Phong shading.
    \item Simple transformations on distance functions such as translation,
        rotation, scaling, and boolean operators.
    \item Advanced transformations on distance functions like infinite
        repetition, skewing, and deformations.
    \item Basic ambient occlusion (AO).
    \item Hard and soft shadows.
    \item The terrain marching algorithm and related optimizations.
    \item Using Perlin noise to generate interesting terrain geometry and
        textures.
    \item Fog and sun rendering.
    \item High-detail rendering of the 2D Mandelbrot fractal using distance
        functions.
    \item Rendering of the 3D Mandelbulb fractal using distance functions.
\end{itemize}

The website is permanently available at

\qquad\url{http://mcjohnalds.github.io/learn-ray-marching/RayMarchingOverview}

Please make sure you are using the latest version of Chrome or Firefox when
viewing it. The code is viewable at

\qquad\url{https://github.com/mcjohnalds/learn-ray-marching/tree/gh-pages}.

\section{The Website}

Each section of the website discusses a technique then follows with a 3D scene
rendered inside the user's browser. The user can click a button to open an
editor to view and edit the code used to render the scene.

The editor contains two panes, and some controls like a play/pause button and
"reset" button. The left pane contains a code editor for editing fragment
shader code, the right pane contains the rendered output of the fragment
shader. (\Cref{editor.png})

Every page on the website contains 3D scenes which are rendered in the browser
using the techniques discussed below.

Integrating a shader editor in the website is a very important feature, because
if the user had to download, install dependencies, compile, and execute all the
code themselves, they would spend far too much time on tasks which offer little
educational value.

The design of the website is optimized for readability, following common
typographic rules.

The website uses pretty graphics, clear code, and interactivity to teach many
computer graphics techniques to anyone who knows the basics of computer
graphics. Unfortunately, writing simple English explanations of the techniques
used, and creating appropriate figures and animations is very time consuming,
so the website is severely lacking in those departments.

\section{Graphics Techniques}

The website covers many ray marching related techniques, explaining how all of
them work and giving functioning implementations that run in the browser. To
aid in the learning process, these functional implementations can be modified
and executed by the user within the website. Ray marches are suitable for
implementing entirely in a fragment shader, so all the techniques following are
implemented solely in fragment shaders.

\subsection{Sphere tracing algorithm}

Ray marchers are a category of rendering algorithms which render 2D or 3D
scenes by marching along rays cast from the viewers' eye position (camera)
until collision with a shape occurs. Shapes are usually defined by distance
functions, which take a point $p$ and return the distance of $p$ from the
boundary of the shape. A ray marching variant known as sphere tracing is given
in the website. (\Cref{shading.png})

\subsection{Phong shading}

Phong shading and its alternatives are not specific to ray marching algorithms
so little space is spent of explaining it. The website uses point lights (like
a light bulb) and directional lights (like the sun) in combination with Phong
shading. (\Cref{shading.png})

\subsection{Distance function transformations}

By applying operations to the input and output of distance functions, many
different shapes can be created with a small amount of code. The website gives
many distance functions for simple shapes and transformations that can be used
to combine or modify shapes in interesting ways. (\Cref{shading.png},
\Cref{infinite-spheres.png})

\subsection{Basic ambient occlusion}

AO simulates the process of light bouncing many times around a scene so
surfaces may be lit even if there is no light directly shining on them. Sphere
tracing algorithms can use the number of marches required to reach a surface to
create a simple approximation of AO. This simple variant of AO is given in the
website. (\Cref{mandelbulb})

\subsection{Soft shadows}

In real life, the edge of a shadow doesn't disappear instantly, but gradually,
creating a penumbra around the darkest part of the shadow. Both soft and hard
shadow implementations are given in the website. (\Cref{shading.png})

\subsection{Terrain marching and Perlin noise}

Functions of the form $y=f(x,z)$ can be rendered using a different type of ray
marching algorithm. The height $y$ and normal to the point $(x,y,z)$ can be
used to create realistic terrain texturing.

Perlin noise allows for natural looking randomly generated terrain geometry and
texture. An implementation for procedural terrain geometry and textures created
from Perlin noise is given in the website. (\Cref{sinxcosz.png}, \Cref{terrain-marching.png})

\subsection{Fog and sun}

Fog and a 2D sun in the sky add a final touch of realism to a scene. A fog
implementation is given in the website. (\Cref{terrain-marching.png})

\subsection{The Mandelbrot fractal}

The Mandelbrot fractal is generated by visualizing all values of $c$ for which
the sequence
\[
    z_{n+1}=z_n^2+c
\]
converges as $n\to\infty$, where $c$ and $z$ are complex numbers. Two
implementations are given: a naive escape time algorithm, and a better
algorithm based on distance functions. (\Cref{mandelbrot})

\subsection{The Mandelbulb fractal}

The Mandelbulb fractal is a version of the Mandelbrot fractal generalized to
three dimensions. A distance function for rendering Mandelbulb fractals is
given in the website. (\Cref{mandelbulb})

\section{Results}

\begin{figure}[H]
    \graphic{shading.png}
    \caption{Demonstration of distance function transformations. Also note the
        soft shadows.}
    \label{shading.png}
\end{figure}

\begin{figure}[H]
    \graphic{infinite-spheres.png}
    \caption{A demonstration of how the \texttt{mod} function can be used to
        create repetition. The shader can be unpaused to enable animation.}
    \label{infinite-spheres.png}
\end{figure}

\begin{figure}[H]
    \graphic{sinxcosz.png}
    \caption{The result of rendering the function $y=\sin(x)\cos(z)$ using a
        terrain marching algorithm.}
    \label{sinxcosz.png}
\end{figure}

\begin{figure}[H]
    \graphic{terrain-marching.png}
    \caption{Demonstrates terrain marching, Perlin noise, sun, and fog.}
    \label{terrain-marching.png}
\end{figure}

\begin{figure}[H]
    \graphic{mandelbrot-escape-time.png}
    \graphic{mandelbrot-dist-func.png}
    \caption{The result of rendering the Mandelbrot fractal with the naive
        escape time algorithm (top) versus using distance functions (bottom).}
    \label{mandelbrot}
\end{figure}

\begin{figure}[H]
    \graphic{mandelbulb2.png}
    \graphic{mandelbulb8.png}
    \caption{Two variants of the Mandelbulb fractal. The shape can be rotated
        and zoomed. Both of these shaders make use of the AO mentioned
        previously.}
    \label{mandelbulb}
\end{figure}

\begin{figure}[H]
    \graphic{editor.png}
    \caption{A screenshot of the shader editor.}
    \label{editor.png}
\end{figure}

\section{Conclusion}

The combination of simple English, maths, and code to explain techniques
ensures that even if a reader fails to understand half of the content they read
through, they will still learn much from the experience.

\section{Disclaimer and Acknowledgments}

I did not invent any of the graphics techniques used, so appropriate references
are in the bibliography, nor did I create many tools the website relies upon:

\begin{itemize}
    \item HTML, CSS, and Javascript: Basic building blocks of any website.
    \item WebGL: Javascript API for hardware accelerated rendering in the
        browser, based on OpenGL ES.
    \item GitHub Pages: Free static website hosting.
    \item Jekyll: Static site generator that integrates with GitHub pages.
    \item ACE editor: A code editor that can be embedded into a webpage.
    \item MathJax: Displays Latex formulas in the browser.
    \item jQuery: A general purpose Javascript utility library.
\end{itemize}

Other than that, all the code and content on the website and in this report is
my own creation.

\nocite{*}
\printbibliography

\end{document}
