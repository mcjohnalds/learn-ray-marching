\documentclass[parskip=full]{scrartcl}
\usepackage{graphicx}
\usepackage{csquotes}
\usepackage{hyperref}
\usepackage{float}
\usepackage[backend=bibtex]{biblatex}

\hypersetup{colorlinks,urlcolor=blue}
\addbibresource{references.bib}
\MakeOuterQuote{"}
\newcommand\graphic[1]{\includegraphics[keepaspectratio=true,width=\textwidth]{../img/#1}}

\begin{document}

\section{Problem Statement}

The project is a website that teaches various basic and advanced techniques
based around the rendering technique known as ray marching. Using maths,
graphics, code snippets, user-modifiable shader programs, and plain English,
the website explains the theory and implementation of:
\begin{itemize}
    \item The sphere tracing algorithm.
    \item Phong shading.
    \item Distance functions.
    \item Simple transformations on distance functions such as translation,
        rotation, scaling, and boolean operators.
    \item Advanced transformations on distance functions like infinite
        repetition, skewing, and deformations.
    \item Basic ambient occlusion.
    \item Hard and soft shadows.
    \item The terrain marching algorithm.
    \item Using Perlin noise to generate interesting terrain geometry and
        textures.
    \item Fog and sun rendering.
    \item High-detail Rendering the 2D Mandelbrot fractal using a distance
        function.
    \item Rendering of the 3D Mandelbulb fractal using a distance function.
\end{itemize}

The website is permanently available at

\qquad\url{http://mcjohnalds.github.io/learn-ray-marching/RayMarchingOverview}.

The code is viewable at

\qquad\url{https://github.com/mcjohnalds/learn-ray-marching}.

\section{Website Implementation}

The website implementation relies on many technologies and languages. Primarily
it uses:
\begin{itemize}
    \item HTML, CSS, and Javascript: Basic building blocks of any website.
    \item WebGL: Javascript API for hardware accelerated rendering in the
        browser, based on OpenGL ES.
    \item GitHub Pages: Free static website hosting.
    \item Jekyll: Static site generator that integrates with GitHub pages.
    \item ACE editor: A code editor that can be embedded into a webpage.
    \item MathJax: Displays latex formulas in the browser.
\end{itemize}

Each section of the website discusses a technique then follows with a 3D scene
rendered inside the user's browser. The user can click a button to open an
editor to view and edit the code used to render the scene.

The editor contains two panes, and some controls like a play/pause button and
"reset" button. The left pane contains a code editor for editing fragment
shader code, the right pane contains the rendered output of the fragment
shader.

Every page on the website contains 3D scenes which are rendered in the browser
using the techniques discussed.

\section{Graphics Techniques}

The website covers many ray marching related techniques, explaining how all of
them work and giving functioning implementations that run in the browser. To
aid in the learning process, these functional implementations can be modified
and executed by the user within the website. Ray marches are suitable for
implementing entirely in a fragment shader, so all the techniques following are
implemented solely in fragment shaders.

\subsection{Sphere tracing algorithm}

Ray marchers are a category of rendering algorithms which render 2D or 3D
scenes by marching along rays cast from the viewers' eye position (camera)
until collision with a shape occurs. Shapes are usually defined by distance
functions, which take a point $p$ and return the distance of $p$ from the
boundary of the shape. A ray marching variant known as sphere tracing is given
in the website.

\subsection{Phong shading}

Phong shading is a crude but effective approximation of ambient, diffuse, and
specular lighting. Phong shading and its alternatives are not specific to
ray marching algorithms so little space is spent of explaining it. The website
uses point lights (like a light bulb) and directional lights (like the sun) in
combination with Phong shading.

\subsection{Distance function transformations}

By applying operations to the input and output of distance functions, many
different shapes can be created with a small amount of code. The website
gives many distance functions for simple shapes and transformations that can
be used to combine or modify shapes in interesting ways.

\subsection{Basic ambient occlusion}

Ambient occlusion simulates the process of light bouncing many times around a
scene so surfaces may be lit even if there is no light directly shining on
them. True ambient occlusion is highly computationally expensive and
complicated to implement, however, one of the perks of ray marching algorithms
is that they output the number of marches it took to reach a surface,
information that can be used to give a simple approximation of ambient
occlusion. This simple variant of ambient occlusion is given in the website.

\subsection{Soft shadows}

In real life, the edge of a shadow doesn't disappear instantly, but gradually,
creating a penumbra around the darkest part of the shadow. Both soft and hard
shadows are given in the website.

\subsection{Terrain marching with Perlin noise}

Functions of the form $y=f(x,z)$ can be used to define the height of terrain
for every $(x,z)$ value, allowing for procedural creation of terrain geometry.
These functions require a different variant of ray marching algorithm to
render.

There are many techniques to texture this terrain. Given a point on the terrain
$(x,y,z)$, the height $y$ and the normal to the surface at $(x,y,z)$ can be
used to decide the color.

Perlin noise allows for natural looking randomly generated terrain geometry and
texture. Procedural terrain geometry and textures created from Perlin noise is
given in the website.

\subsection{Fog and sun}

Fog and a 2D sun in the sky add a final touch of realism to a scene.
An exponential fog model is given in the website \cite{microsoft-fog-formulas}.

\subsection{The Mandelbrot fractal}

The Mandelbrot fractal is generated by visualizing all values of $c$ for which
\[
    z_{n+1}=z_n^2+c
\]
converges as $n\to\infty$, where $c$ and $z$ are complex numbers. Two
implementations are given: a naive escape time algorithm, and a better
algorithm based on distance functions.

\subsection{The Mandelbulb fractal}

The Mandelbulb fractal is a version of the Mandelbrot fractal generalized to
three dimensions. A distance function for rendering Mandelbulb fractals is
given in the website.

\section{Results}

\begin{figure}[H]
    \graphic{shading.png}
    \caption{Demonstration of distance function transformations. In the back
        row from left to right: A union operation between a sphere and a cube,
        a difference operation, and an intersection operation. In the front
        row from left to right: A displacement function applied to a cylinder,
        a skewed cube, and a very twisted cylinder. Also note the soft
        shadows.}
    \bigskip
    \graphic{infinite-spheres.png}
    \caption{A demonstration of the \texttt{mod} function can be used to create
        infinite repetition without a large computational cost. In the code
        editor, the shader can be unpaused to enable animation.}
\end{figure}

\begin{figure}[H]
    \graphic{sinxcosz.png}
    \caption{The result of rendering the function $y=\sin(x)\cos(z)$ using a
        terrain marching algorithm.}
    \bigskip
    \graphic{terrain-marching.png}
    \caption{The result of rendering a function based on Perlin noise using a
        terrain marching algorithm. The terrain is textured by taking into
        account the height and normal of the surface being textured. Also
        includes a sun and fog.}
\end{figure}

\begin{figure}[H]
    \graphic{mandelbrot-escape-time.png}
    \graphic{mandelbrot-dist-func.png}
    \caption{The result of rendering the Mandelbrot fractal with the naive
        escape time algorithm versus using distance functions.}
\end{figure}

\begin{figure}[H]
    \graphic{mandelbulb2.png}
    \graphic{mandelbulb8.png}
    \caption{On the top: A ray marched Mandelbulb fractal given by
        $z_{n+1}=z_n^2+c$ where $z$ and $c$ are triplex numbers (complex
        numbers with 3 dimensions). On the bottom: The same, but with
        $z_{n+1}=z_n^8+c$. In the website, the bottom image is an animated and
        interactive shader. When paused, the fractal can be rotated with the
        mouse. When unpaused, the shader is constantly zooming in to the
        fractal, revealing the never-ending complexity the shader provides.
        Both of these shaders make use of the cheap ambient occlusion mentioned
        previously.}
\end{figure}

\begin{figure}[H]
    \graphic{editor.png}
    \caption{A screenshot of the shader editor.}
\end{figure}

\section{Discussion}

The combination of simple English, maths, and code to explain techniques
ensures that even if a reader fails to understand half of the content they read
through, they will still learn much from the experience.

Integrating a shader editor in the website is a very important feature, because
if the user had to download, install dependencies, compile, and execute all the
code themselves, they would spend far too much time on tasks which offer little
educational value.

The design of the website is optimized for readability, following common
typographic rules.

The website uses pretty graphics, clear code, and interactivity to teach many
computer graphics techniques to anyone who knows the basics of computer
graphics. Unfortunately, writing simple English explanations of the techniques
used, and creating appropriate figures and animations is very time consuming,
so the website is severely lacking in those departments.

\nocite{*}
\printbibliography

\end{document}
